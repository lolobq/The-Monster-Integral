\documentclass[12 pt]{article}
\usepackage[utf8]{inputenc}

\title{The Monster}
\author{Lauren Bourque }
\date{September 16, 2018}

\begin{document}

\maketitle

\section{The Problem}
$$\int_{}^{} \frac{3}{x^3-1} dx$$
\section{Breaking up and Solving Partial Fractions}
$$\frac{3}{(x-1)(x^2+x+1)}=\frac{a}{x-1}+\frac{bx+c}{x^2+x+1}$$\\
The denominator of the integral can be broken up into $(x-1)(x^2+x+1)$ as shown. The fraction is written by using partial fractions and set equal to the partial fractions.\\\\
$$[(x-1)(x^2+x+1)](\frac{3}{(x-1)(x^2+x+1)}=\frac{a}{x-1}+\frac{bx+c}{x^2+x+1})$$\\
The denominator of the fraction on the left is multiplied by both sides of the equation to solve the equation.\\\\
$$3=a(x^2+x+1)+(bx+c)(x-1)$$\\
After the denominator is multiplied across the equation, this is what remains.\\\\
$when x=1,\\
3=a(3)$\\
$when x=1\\
a=1$\\
Set x=1 so that a can easily be solved for.\\\\
$0x^2=ax^2+bx^2$\\
$0=a+b$\\
$0=1+b$\\
$-1=b$\\
1 was substituted in for a when solving $0x^2=ax^2+bx^2$\\\\
$3=a-c$\\
$3=1-c$\\
$-2=c$\\
1 was substituted in for a when solving $3=a-c$\\\\
\section{Return to the Integral and Solve for the First Part}
$$\int_{}^{} \frac{1}{x-1}+\frac{-x-2}{x^2+x+1} dx$$\\
Substitute the values for a, b, and c into the partial fractions and replace the initial integrand with the new integrand with the values for a, b, and c.\\\\
$$\int_{}^{} \frac{1}{x-1} dx+ \int_{}^{} \frac{-x-2}{x^2+x+1} dx$$
$$\int_{}^{} \frac{1}{x-1} dx= ln \mid(x-1)\mid$$\\
Find the antiderivative for the first part of the integral. Remember this for later.\\\\
\section{Return to the Second Part of the Integral}
$$\int_{}^{} \frac{-x-2}{x^2+x+1} dx (x^2+x+1)=u, 2x+1dx=du$$\\
Use u-substitution to help solve the integral.\\\\
$$(-2)\frac{-1}{2}\int_{}^{} \frac{-x-2}{x^2+x+1} dx$$\\
Multiply the outside by -2 and -1/2 to try to get the integral to a state where u-substitution can be used.\\\\
$$\frac{-1}{2}\int_{}^{} \frac{2x+4}{x^2+x+1} dx$$\\
$$\frac{-1}{2}\int_{}^{} \frac{2x+4}{x^2+x+1}+ \frac{3}{x^2+2+1} dx$$\\
Break up $\frac{2x+4}{x^2+x+1}$ into two separate fractions to be able to use u-substitution.\\\\
$$\frac{-1}{2}\int_{}^{} \frac{2x+4}{x^2+x+1} dx- \frac{1}{2}(3)\int_{}^{} \frac{1}{x^2+2+1} dx$$\\
Separate into 2 integrals.\\\\
\section{Rewrite the Denominator of the Second Integral}
$$(x^2+x+1)=(x^2+x+\frac{1}{4})+1-\frac{1}{4}$$\\
$$(x+\frac{1}{2})^2+\frac{3}{4}$$\\
$$(x+\frac{1}{2})^2+(\frac{\sqrt{3}}{2})^2$$\\
Rewrite the denominator so that it can be solve with a formula.
\section{Return to the Integral}
$-\frac{1}{2}\int_{}^{} \frac{1}{u}du-\frac{3}{2}(\frac{2}{\sqrt{3}})
arctan[\frac{\sqrt{3}}{2}(x+\frac{1}{2})+c]$\\
$-\frac{1}{2}ln\mid u\mid-\frac{3}{2}(\frac{2\sqrt{3}}{3})arctan[\frac{\sqrt{3}}{2}(x+\frac{1}{2})+c]$\\
$-\frac{1}{2}ln\mid x^2+x+1\mid-\sqrt{3}arctan[\frac{\sqrt{3}}{2}(x+\frac{1}{2})+c]$\\\\
$ln\mid x-1\mid-\frac{1}{2}ln\mid x^2+x+1\mid-\sqrt{3}arctan[\frac{\sqrt{3}}{2}(x+\frac{1}{2})+c]+c$\\
Write the answer, including the first part of the integral.
\end{document}
